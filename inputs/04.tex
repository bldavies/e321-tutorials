% 04.tex
% Benjamin Davies
% 2017 03 11

\begin{enumerate}

	\item
	Suppose that a consumer has utility function
	\[ u(x)=\ln(x_1x_2), \]
	where~$x_i\ge1$ is the quantity demanded for each good~$i\in\{1,2\}$.
	%
	\begin{enumerate}

		\item
		Use Jensen's inequality to show that~$u(x)$ is concave.
		%
		\begin{solution}
			First notice that the second derivative
			\[ \der{^2}{t^2}\ln(t)=-\frac{1}{t^2} \]
			is negative whenever~$t\ge1$.
			Hence the natural logarithm is concave and therefore
			\[ \ln(\lambda t_1+(1-\lambda)t_2)%
				\ge\lambda\ln(t_1)+(1-\lambda)\ln(t_2) \]
			for all~$\lambda\in[0,1]$ and~$t_1,t_2\ge1$ by Jensen's inequality.
			Now, define the set
			\[ X=\{x\in\R^2:x_1\ge1\ \text{and}\ x_2\ge1\} \]
			and let~$x^1,x^2\in X$ be arbitrary.
			Then for all~$\lambda\in[0,1]$ we have
			%
			\begin{align}
				u(\lambda x^1+(1-\lambda)x^2)
				&= \ln((\lambda x_1^1+(1-\lambda)x_1^2)%
					(\lambda x_2^1+(1-\lambda)x_2^2))\\
				&= \ln(\lambda x_1^1+(1-\lambda)x_1^2)\\
					&\quad+\ln(\lambda x_2^1+(1-\lambda)x_2^2)\\
				&\ge \lambda\ln(x_1^1)+(1-\lambda)\ln(x_1^2)\\
					&\quad+\lambda\ln(x_2^1)+(1-\lambda)\ln(x_2^2)\\
				&= \lambda\ln(x_1^1x_2^1)+(1-\lambda)\ln(x_2^1x_2^2)\\
				&= \lambda u(x^1)+(1-\lambda)u(x^2).
			\end{align}
			%
			It follows that~$u(x)$ is concave by Jensen's inequality.
		\end{solution}

		\item
		Show that~$u(x)$ has convex contours.
		%
		\begin{solution}
			Let~$c\ge0$ be arbitrary and consider the contour defined by~$u(x)=c$.
			We can write
			%
			\begin{align}
				\exp(c)
				&= \exp(\ln(x_1x_2))\\
				&= x_1x_2
			\end{align}
			%
			because the exponential and natural logarithm are inverse functions.
			Dividing both sides by~$x_1$ gives
			\[ x_2=f(x_1), \]
			where the function
			\[ f(x_1)=\frac{c}{x_1}. \]
			Then the second derivative
			\[ f''(x_1)=\frac{2c}{x_1^3} \]
			is positive because~$x_1\ge1$ and~$c\ge0$.
			It follows that~$u(x)$ has convex contours.
		\end{solution}

		\item
		Find an expression for the consumer's marginal rate of substitution as a function of~$x_1$ and~$x_2$.
		%
		\begin{solution}
			By the implicit function theorem, we have
			%
			\begin{align}
				\der{x_2}{x_2}\bigg\vert_{\d u=0}
				&= -\frac{\p u(x)/\p x_1}{\p u(x)/\p x_2}\\
				&= -\frac{1/x_1}{1/x_2}\\
				&= -\frac{x_2}{x_1}
			\end{align}
			%
			so that the consumer's marginal rate of substitution
			\[ \text{MRS}(x)=-\frac{x_2}{x_1}. \]
		\end{solution}

		\item
		Show that the consumer optimally spends the same amount of wealth on each good.
		%
		\begin{solution}
			The optimal demand~$x_i^*$ for each good~$i\in\{1,2\}$ satisfies the tangency condition
			%
			\begin{align}
				-\frac{p_1}{p_2}
				&= \text{MRS}(x^*)\\
				&= -\frac{x_2^*}{x_1^*},
			\end{align}
			%
			which implies that~$p_1x_1^*=p_2x_2^*$.
			Hence the consumer optimally spends the same amount of wealth on each good.
		\end{solution}

	\end{enumerate}

	\item
	Suppose that a consumer has utility function
	\[ u(x)=x_1^\alpha x_2^\beta, \]
	where~$x_i>0$ is the quantity demanded for each good~$i\in\{1,2\}$, and~$\alpha$ and~$\beta$ are positive constants.
	%
	\begin{enumerate}
		
		\item
		Find an expression for the consumer's marginal rate of substitution as a function of~$x_1$ and~$x_2$.
		%
		\begin{solution}
			The partial derivatives
			\[ \pder{u(x)}{x_1}=\alpha x_1^{\alpha-1}x_2^\beta \]
			and
			\[ \pder{u(x)}{x_2}=\beta x_1^\alpha x_2^{\beta-1}. \]
			By the implicit function theorem, we have
			%
			\begin{align}
				\der{x_2}{x_1}\bigg\vert_{\d u=0}
				&= -\frac{\p u(x)/\p x_1}{\p u(x)/\p x_2}\\
				&= -\frac{\alpha x_1^{\alpha-1}x_2^\beta}%
					{\beta x_1^\alpha x_2^{\beta-1}}\\
				&= -\frac{\alpha x_2}{\beta x_1}
			\end{align}
			%
			so that the consumer's marginal rate of substitution
			\[ \text{MRS}(x)=-\frac{\alpha x_2}{\beta x_1}. \]
		\end{solution}
		
		\item
		Show that the optimal expenditure ratio
		\[ \frac{p_1x_1^*}{p_2x_2^*}=\frac{\alpha}{\beta}. \label{eq:cobb_b_goal} \]
		%
		\begin{solution}
			The optimal demand~$x_i^*$ for each good~$i\in\{1,2\}$ satisfies the tangency condition
			%
			\begin{align}
				-\frac{p_1}{p_2}
				&= \text{MRS}(x^*)\\
				&= -\frac{\alpha x_2^*}{\beta x_1^*}
			\end{align}
			%
			which can be rewritten as
			\[ \frac{p_1x_1^*}{p_2x_2^*}=\frac{\alpha}{\beta}. \]
		\end{solution}
		
		\item
		Use~\eqref{eq:cobb_b_goal} and the budget constraint
		\[ p_1x_1^*+p_2x_2^*=w \]
		to show that
		\[ \pder{x_1^*}{\alpha}=\frac{p_2x_2^*}{(\alpha+\beta)p_1}. \]
		%
		\begin{solution}
			First write the tangency condition as
			\[ \beta p_1x_1^*=\alpha p_2x_2^*. \]
			Differentiating both sides with respect to~$\alpha$ gives
			\[ \beta p_1\pder{x_1^*}{\alpha}%
				=p_2x_2^*+\alpha p_2\pder{x_2^*}{\alpha}
				\label{eq:cobb_c_der} \]
			by the product rule.
			We can also differentiate the budget constraint with respect to~$\alpha$ to obtain
			\[ p_1\pder{x_1^*}{\alpha}+p_2\pder{x_2^*}{\alpha}=0 \]
			so that
			\[ \pder{x_2^*}{\alpha}=-\frac{p_1}{p_2}\pder{x_1^*}{\alpha}. \]
			Substituting this latter expression into~\eqref{eq:cobb_c_der} gives
			\[ \beta p_1\pder{x_1^*}{\alpha}%
				=p_2x_2^*-\alpha p_1\pder{x_1^*}{\alpha}, \]
			which we can rearrange to obtain
			\[ \pder{x_1^*}{\alpha}=\frac{p_2x_2^*}{(\alpha+\beta)p_1}. \]
		\end{solution}
		
	\end{enumerate}

\end{enumerate}
