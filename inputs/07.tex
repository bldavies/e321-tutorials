% 07.tex
% Benjamin Davies
% 2017 04 04

\begin{enumerate}

	\item
	Suppose that a consumer solves
	\[ \max_xu(x)=\sqrt{x_1x_2}\ \text{subject to}\ x_1+2x_2\le w, \]
	where~$w>0$ is the consumer's disposable income and~$x_i\ge0$ is quantity demanded of each good~$i\in\{1,2\}$.
	%
	\begin{enumerate}

		\item
		Write down the Lagrangian for the consumer's problem.
		Ignore any nonnegativity constraints.
		%
		\begin{solution}
			The Lagrangian is given by
			\[ \Lcal(x,\delta)=\sqrt{x_1x_2}+\delta(w-x_1-2x_2), \]
			where~$\delta$ is a Lagrange multiplier.
		\end{solution}

		\item
		Derive the first-order and complementary slackness conditions for an optimal solution to the consumer's problem.
		%
		\begin{solution}
			The first-order conditions for a maximum are
			%
			\begin{align}
				0
				&= \pder{\Lcal(x^*,\delta)}{x_1}\\
				&= \frac{1}{2}\sqrt{\frac{x_2^*}{x_1^*}}-\delta 
					\label{eq:sqrt_foc1}\\
				0
				&= \pder{\Lcal(x^*,\delta)}{x_2}\\
				&= \frac{1}{2}\sqrt{\frac{x_1^*}{x_2^*}}-2\delta
					\label{eq:sqrt_foc2}
			\end{align}
			%
			and the complementary slackness condition is
			\[ \delta(w-x_1^*-2x_2^*)=0. \]
		\end{solution}

		\item
		Find the optimal solution to the consumer's problem.
		Interpret the value of the Lagrange multiplier for the budget constraint.
		%
		\begin{solution}
			Substituting~\eqref{eq:sqrt_foc1} into~\eqref{eq:sqrt_foc2} so as to eliminate~$\delta$ gives
			\[ \sqrt{\frac{x_2^*}{x_1^*}}%
				=\frac{1}{2}\sqrt{\frac{x_1^*}{x_2^*}}, \]
			which implies that~$x_1^*=2x_2^*$.
			The Lagrange multipler
			%
			\begin{align}
				\delta
				&= \frac{1}{2}\sqrt{\frac{x_2^*}{2x_2^*}}\\
				&= \frac{1}{2\sqrt{2}}
			\end{align}
			%
			is strictly positive and so the budget constaint is binding.
			Hence
			%
			\begin{align}
				w
				&= x_1^*+2x_2^*\\
				&= 2x_1^*
			\end{align}
			%
			so that~$x^*=(w/2,w/4)$.
			The Lagrange multiplier value implies that an increase in wealth of~\$1 would generate an additional~$1/2\sqrt{2}$ units of utility.
		\end{solution}

		\item
		Show that the consumer's indirect utility function is linear in~$w$.
		%
		\begin{solution}
			The consumer's objective function is given by
			%
			\begin{align}
				v(w)
				&= u(x^*)\\
				&= \sqrt{\frac{w^2}{8}}\\
				&= \frac{w}{2\sqrt{2}}
			\end{align}
			%
			The second derivative
			\[ v''(w)=0 \]
			is nonnegative and therefore~$v(w)$ is linear.
		\end{solution}

		\item
		Suppose that the budget constraint changes to~$x_1+3x_2\le w$.
		Without any calculations, explain what will happen to~$x_1^*$.
		%
		\begin{solution}
			For Cobb-Douglas utility functions with~$\alpha=\beta$, the consumer spends the same amount of wealth on each good.
			But the price of good~1 hasn't changed and so the optimal demand for good~1 will remain at~$x_1^*=w/2$.
		\end{solution}

	\end{enumerate}

	\item
	Consider the constrained minimisation problem
	\[ \min_x v_1x^2+v_2(1-x)^2+2cx(1-x)\ \text{subject to}\ r_1x+r_2(1-x)\ge R, \]
	where~$v_1$, $v_2$, $c$, $r_1$, $r_2$ and~$R$ are positive constants, and~$v_1+v_2>2c$.
	%
	\begin{enumerate}

		\item
		Write down the Lagrangian for this problem.
		%
		\begin{solution}
			The Lagrangian is given by
			\[ \Lcal(x,\delta)%
				=-v_1x^2-v_2(1-x)^2-2cx(1-x)%
				+\delta(r_1x+r_2(1-x)-R), \]
			where~$\delta\ge0$ is a Lagrange multiplier.
		\end{solution}

		\item
		Show that the Lagrangian from part~a) is maximised by
		% Show that the optimal solution
		\[ x^*%
			=\frac{\delta(r_1-r_2)}{2(v_1+v_2-2c)}%
			+\frac{v_2-c}{v_1+v_2-2c}, \label{eq:two_assets_sol} \]
		where~$\delta$ is the Lagrange multiplier for the inequality constraint.
		%
		\begin{solution}
			The maximiser~$x^*$ satisfies the first-order condition
			%
			\begin{align}
				0
				&= \pder{\Lcal(x^*,\delta)}{x}\\
				&= -2v_1x^*+2v_2(1-x^*)-2c(1-2x^*)+\delta(r_1-r_2)\\
				&= 2v_2-2c+\delta(r_1-r_2)-2x^*(v_1+v_2-2c),
			\end{align}
			%
			which can be rearranged for~\eqref{eq:two_assets_sol}.
			Now
			\[ \pder{^2\Lcal(x^*,\delta)}{x^2}=-2(v_1+v_2-2c) \]
			is strictly negative and therefore~$x^*$ is indeed a maximum.

		\end{solution}

		\item
		Interpret the Lagrange multiplier~$\delta$ from part~b).
		%
		\begin{solution}
			We have
			\[ \pder{\Lcal(x^*,\delta)}{R}=-\delta. \]
			So if~$R$ increases by one unit then the maximal value of the objective function \emph{decreases} by about~$\delta$ units.
		\end{solution}

		\item
		Show that if~$r_1=r_2$ then~$x^*>0$ if and only if~$v_2>c$.
		%
		\begin{solution}
			If~$r_1=r_2$ then the maximiser
			\[ x^*=\frac{v_2-c}{v_1+v_2-2c}. \]
			Now~$v_1+v_2-2c>0$ and therefore~$\sign(x^*)=\sign(v_2-c)$.
		\end{solution}

	\end{enumerate}

\end{enumerate}
