% 05.tex
% Benjamin Davies
% 2017 03 25

\begin{enumerate}

	\item
	Suppose that market demand for a good is given by the linear function
	\[ x(p)=b-ap, \]
	where~$p>0$ is the per-unit price of the good and~$a,b>0$ are constants.
	Assume that the market is a monopoly and that the firm produces each unit at a constant cost~$c>0$.
	%
	\begin{enumerate}

		\item
		Write an expression for the firm's profit function~$\pi(p)$ in terms of the price~$p$, and the parameters~$a$, $b$ and~$c$.
		%
		\begin{solution}
			The firm's profit function is given by
			%
			\begin{align}
				\pi(p)
				&= px(p)-cx(p)\\
				&= p(b-ap)-c(b-ap)\\
				&= -ap^2+(b+ac)p-bc,
			\end{align}
			%
			where~$a$, $b$ and~$c$ are treated as parameters.
		\end{solution}

		\item
		Show that the profit function~$\pi(p)$ is concave.
		%
		\begin{solution}
			The second derivative
			\[ \pi''(p)=-2a \]
			is negative because~$a>0$.
			It follows that~$\pi(p)$ is concave.
		\end{solution}

		\item
		Derive the profit-maximising price~$p^*$.
		%
		\begin{solution}
			The profit-maximising price~$p^*$ satisfies the first-order condition
			%
			\begin{align}
				0
				&= \pi'(p^*)\\
				&= -2ap^*+b+ac
			\end{align}
			%
			so that
			\[ p^*=\frac{b+ac}{2a}. \]
			The second-order condition is satisfied because~$\pi(p)$ is concave.
		\end{solution}

		\item
		Find the firm's maximal level of profit~$\pi(p^*)$.
		%
		\begin{solution}
			The firm's maximal level of profit
			%
			\begin{align}
				\pi(p^*)
				&= -a(p^*)^2+(b+ac)p^*-bc\\
				&= -a\bigg(\frac{b+ac}{2a}\bigg)^2%
					+(b+ac)\bigg(\frac{b+ac}{2a}\bigg)%
					-bc\\
				&= \frac{-(b+ac)^2+2(b+ac)^2-4abc}{4a}\\
				&= \frac{(b-ac)^2}{4a}.
			\end{align}
		\end{solution}

	\end{enumerate}

	\item
	Suppose that a tax is imposed on a perfectly competitive market.
	The demand curve is defined by
	\[ p=a-bx, \]
	where~$p>0$ is the market price, $x$~is the quantity demanded and~$a,b>0$ are constants.
	The supply curve is defined by
	\[ p=c+dx+t, \]
	where~$t\ge0$ is the amount of the tax and~$c,d>0$ are constants.
	%
	\begin{enumerate}

		\item
		Find the equilibrium quantity~$x^*$ and price~$p^*$.
		%
		\begin{solution}
			The equilibrium price quantity~$x^*$ satisfies
			\[ a-bx^*=c+dx^*+t, \]
			which has unique solution
			\[ x^*=\frac{a-c-t}{b+d}. \]
			Hence the equilibrium price
			%
			\begin{align}
				p^*
				&= a-bx^*\\
				&= a-b\left(\frac{a-c-t}{b+d}\right)\\
				&= \frac{ad+b(c+t)}{b+d}. 
			\end{align}
		\end{solution}

		\item
		Write an expression for the amount of tax revenue~$r(t)$ collected in terms of the choice variable~$t$, and the parameters~$a$, $b$, $c$ and $d$.
		%
		\begin{solution}
			We have
			%
			\begin{align}
				r(t)
				&= tx^*\\
				&= \frac{(a-c)t-t^2}{b+d}.
			\end{align}
		\end{solution}

		\item
		Solve the constrained maximisation problem
		\[ \max_tr(t)\ \text{subject to}\ t\ge0 \label{eq:tax_c_prob} \]
		for the revenue-maximising amount of tax~$t^*$.
		%
		\begin{solution}
			Assume that~$t^*>0$.
			The second derivative
			\[ r''(t)=-\frac{2}{b+d} \]
			is strictly negative and therefore~$r(t)$ is concave.
			Hence the revenue-maximising tax~$t^*$ satisfies the first-order condition
			%
			\begin{align}
				0
				&= r'(t^*)\\
				&= \frac{(a-c)-2t^*}{b+d}
			\end{align}
			%
			so that
			\[ t^*=\frac{a-c}{2}. \]
			Now~$r(0)=0$ so~$t^*$ is optimal if and only if~$a-c$.
		\end{solution}

		\item
		The solution to~\eqref{eq:tax_c_prob} is strictly positive if and only if~$a>c$.
		Is this condition reasonable?
		%
		\begin{solution}
			Yes; the condition~$a>c$ is a necessary condition for~$x^*$ to be positive.
		\end{solution}

	\end{enumerate}

	\item
	Let~$X$ be a convex set and consider the constrained maximisation problem
	\[ \max_xf(x)\ \text{subject to}\ x\in X, \]
	where~$f(x)$ is increasing and strictly concave.
	Use Jensen's inequality to show that the optimal solution to this problem is unique.
	%
	\begin{solution}
		Assume that~$x^1,x^2\in X$ are distinct optimal solutions.
		Then
		\[ \lambda x^1+(1-\lambda)x^2\in X\]
		for all~$\lambda\in[0,1]$ because~$X$ is a convex set.
		But~$f(x)$ is strictly concave and therefore
		%
		\begin{align}
			f(\lambda x^1+(1-\lambda)x^2)
			&> \lambda f(x^1)+(1-\lambda)f(x^2)\\
			&\ge \min\{f(x^1),f(x^2)\}
		\end{align}
		%
		by Jensen's inequality.
		This contradicts our assumption that~$x^1$ and~$x^2$ are optimal solutions.
		Hence the problem must have a unique solution.
	\end{solution}

\end{enumerate}
